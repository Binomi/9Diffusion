% Für Bindekorrektur als optionales Argument "BCORfaktormitmaßeinheit", dann
% sieht auch Option "twoside" vernünftig aus
% Näheres zu "scrartcl" bzw. "scrreprt" und "scrbook" siehe KOMA-Skript Doku
\documentclass[12pt,a4paper,titlepage,headinclude,bibtotoc]{scrartcl}


%---- Allgemeine Layout Einstellungen ------------------------------------------

% Für Kopf und Fußzeilen, siehe auch KOMA-Skript Doku
\usepackage[komastyle]{scrpage2}
\pagestyle{scrheadings}
\automark[section]{chapter}
\setheadsepline{0.5pt}[\color{black}]

%keine Einrückung
\parindent0pt

%Einstellungen für Figuren- und Tabellenbeschriftungen
\setkomafont{captionlabel}{\sffamily\bfseries}
\setcapindent{0em}


%---- Weitere Pakete -----------------------------------------------------------
% Die Pakete sind alle in der TeX Live Distribution enthalten. Wichtige Adressen
% www.ctan.org, www.dante.de

% Sprachunterstützung
\usepackage[ngerman]{babel}

% Benutzung von Umlauten direkt im Text
% entweder "latin1" oder "utf8"
\usepackage[utf8]{inputenc}

% Pakete mit Mathesymbolen und zur Beseitigung von Schwächen der Mathe-Umgebung
\usepackage{latexsym,exscale,amssymb,amsmath}

% Weitere Symbole
\usepackage[nointegrals]{wasysym}
\usepackage{eurosym}

% Anderes Literaturverzeichnisformat
%\usepackage[square,sort&compress]{natbib}

% Für Farbe
\usepackage{color}

% Zur Graphikausgabe
%Beipiel: \includegraphics[width=\textwidth]{grafik.png}
\usepackage{graphicx}

% Text umfließt Graphiken und Tabellen
% Beispiel:
% \begin{wrapfigure}[Zeilenanzahl]{"l" oder "r"}{breite}
%   \centering
%   \includegraphics[width=...]{grafik}
%   \caption{Beschriftung} 
%   \label{fig:grafik}
% \end{wrapfigure}
\usepackage{wrapfig}

% Mehrere Abbildungen nebeneinander
% Beispiel:
% \begin{figure}[htb]
%   \centering
%   \subfigure[Beschriftung 1\label{fig:label1}]
%   {\includegraphics[width=0.49\textwidth]{grafik1}}
%   \hfill
%   \subfigure[Beschriftung 2\label{fig:label2}]
%   {\includegraphics[width=0.49\textwidth]{grafik2}}
%   \caption{Beschriftung allgemein}
%   \label{fig:label-gesamt}
% \end{figure}
\usepackage{subfigure}

% Caption neben Abbildung
% Beispiel:
% \sidecaptionvpos{figure}{"c" oder "t" oder "b"}
% \begin{SCfigure}[rel. Breite (normalerweise = 1)][hbt]
%   \centering
%   \includegraphics[width=0.5\textwidth]{grafik.png}
%   \caption{Beschreibung}
%   \label{fig:}
% \end{SCfigure}
\usepackage{sidecap}

% Befehl für "Entspricht"-Zeichen
\newcommand{\corresponds}{\ensuremath{\mathrel{\widehat{=}}}}

%Für chemische Formeln (von www.dante.de)
%% Anpassung an LaTeX(2e) von Bernd Raichle
\makeatletter
\DeclareRobustCommand{\chemical}[1]{%
  {\(\m@th
   \edef\resetfontdimens{\noexpand\)%
       \fontdimen16\textfont2=\the\fontdimen16\textfont2
       \fontdimen17\textfont2=\the\fontdimen17\textfont2\relax}%
   \fontdimen16\textfont2=2.7pt \fontdimen17\textfont2=2.7pt
   \mathrm{#1}%
   \resetfontdimens}}
\makeatother

%Si Einheiten
\usepackage{siunitx}

%c++ Code einbinden
\usepackage{listings}
\lstset{numbers=left, numberstyle=\tiny, numbersep=5pt}

%errorFkt
\newcommand{\erf}{\ensuremath{\text{erf}}}

%Differential
\newcommand{\dif}{\ensuremath{\mathrm{d}}}

%Boxen,etc.
\usepackage{fancybox}
\usepackage{empheq}

\begin{document}

\begin{titlepage}
\centering
\textsc{\Large Anfängerpraktikum der Fakultät für
  Physik,\\[1.5ex] Universität Göttingen}

\vspace*{4.2cm}

\rule{\textwidth}{1pt}\\[0.5cm]
{\huge \bfseries
  Diffusion\\[1.5ex]
  Protokoll:}\\[0.5cm]
\rule{\textwidth}{1pt}

\vspace*{3.0cm}

\begin{Large}
\begin{tabular}{ll}
Praktikant:
%	&  Skrollan Detzler\\
 	&  Felix Kurtz\\
% 	&  Michael Lohmann\\
%	&  Kevin Lüdemann\\

  E-Mail: 
%	&  skrollan.detzler@stud.uni-goettingen.de\\
	&  felix.kurtz@stud.uni-goettingen.de\\
%	& m.lohmann@stud.uni-goettingen.de\\
%	&  kevin.luedemann@stud.uni-goettingen.de\\

Versuchspartner:
	&  Skrollan Detzler\\
% 	&  Felix Kurtz\\
% 	&  Michael Lohmann\\
%	&  Kevin Lüdemann\\

 Betreuer: & Martin Ochmann\\
 Versuchsdatum: & 30.06.2014\\
\end{tabular}
\end{Large}

\vspace*{0.8cm}

\begin{Large}
\fbox{
  \begin{minipage}[t][2.5cm][t]{6cm} 
    Note:
  \end{minipage}
}
\end{Large}

\end{titlepage}

\tableofcontents

\newpage

\section{Einleitung}
\label{sec:einleitung}
In diesem Versuch soll das Phänomen der \textit{Diffusion} untersucht werden.
Darunter versteht man die Durchmischung von zwei verschiedenen Gasen oder Flüssigkeiten, welche mit der Zeit vonstatten geht.
Sie spielt besonders in der Biologie bei osmotischen Prozessen eine große Rolle.
Als eine von vielen Transportphänomenen wie Wärmeleitung ist sie jedoch am besten experimentell messbar.\\
Wir wollen hier die Diffusion von Methylenblau in Wasser untersuchen.

\section{Theorie}
\label{sec:theorie}
\subsection{Brown'sche Bewegung}
Den Effekt der Diffusion kann man qualitativ durch die \textit{Brown'sche Molekularbewegung} erklären.
So bewegen sich die Moleküle in einer Flüssigkeit aufgrund ihrer Temperatur und stoßen mit anderen Teilchen zusammen.
Dabei vermischen sich zwei verschiedene Flüssigkeiten.\\
Man kann diese Bewegung durch Staubkörner sichtbar machen.
Diese sind nämlich um Größenordnungen größer als die Flüssigkeitsmoleküle und können so mit dem Mikroskop gesehen werden.

\subsection{Ficksche Gesetze}
Man kann die Diffusion allerdings auch quantitativ beschreiben.
So erhält man die von A.Fick  im Jahre 1855 auf empirischer Basis aufgestellten und nach ihm benannten Gesetze.
Diese wurden später von  Albert Einstein aus der Thermodynamik abgeleitet.\\
%Quelle: wiki

\textbf{1.Ficksches Gesetz}\\
Die Teilchen wollen die Inhomogenität ausgleichen und fließen so zu Orten geringer Konzentration. Mit der Teilchenstromdichte $\vec j$, dem Konzentrationsgradienten grad $c$ und  dem Diffusionskoeffizienten $D$, der hier orts- und zeitunabhängig sein soll, ergibt sich das 1.Ficksche Gesetz.
\begin{empheq}[box=\shadowbox*]{align}
\vec{j}(\vec{x})=-D\cdot\nabla c
\end{empheq}

\textbf{2.Ficksches Gesetz}\\
Da die Teilchenzahl konstant ist, ergibt sich die Kontinuitätsgleichung $\frac{\partial c}{\partial t} = - \nabla \cdot\vec{j}$.
Kombiniert man dies mit dem 1.Gesetz erhält man das zweite Gesetz.

\begin{empheq}[box=\shadowbox*]{align}
\frac{\partial n}{\partial t}=-D\cdot\Delta c
\end{empheq}

In unserem Versuchsaufbau befindet sich die Grenzschicht zwischen den beiden Flüssigkeiten bei $x=0$.
\begin{align*}
	c(x,t=0)=\left\{\begin{array}{ll} c_0, & x \leq 0 \\
         0, & x>0\end{array}\right. 
\end{align*}

Man kann das 2.Ficksche Gesetz für diese Anfangsbedingung analytisch lösen. Dabei ergibt sich diese Lösung:
\begin{align}
	c(x,t)=\frac{c_0}{2} \left[1-\erf\left(\frac{x}{\sqrt{4Dt}}\right)\right]	
	\label{eq:DiffLsg}
\end{align}
mit der Gaußschen Fehlerfunktion, die so definiert ist
\begin{align*}
	\erf(y):=\frac{2}{\sqrt{\pi}} \int_0^y \! e^{-v^2}\, \dif v ~.
\end{align*}

\subsection{Wheatstone'sche Messbrücke und Fotowiderstand}
Diese Schaltung (siehe Abb. \ref{fig:bruecke})  wird verwendet, um einen unbekannten Widerstand  oder kleine Widerstandsänderungen zu messen.
In unserem Fall ist letzteres der Fall.
Misst man keine Spannungsdifferenz, gleichbedeutend mit keinem Stromfluss, dann gilt
\begin{align*}
	\frac{R_1}{R_2}=\frac{R_3}{R_4}
\end{align*}
Bei dem Aufbau, der in diesem Versuch vorliegt, kann man 3 verschiedene Widerstände für $R_1$ wählen, während $R_2$ ein Fotowiderstand ist.
$R_3$ und $R_4$ bilden ein Potentiometer, welches in der Mitte frei abgegriffen werden kann.\\
Je mehr Licht auf den \textit{Fotowiderstand} fällt, desto kleiner ist sein ohmscher Widerstand.
Dies beruht auf dem inneren fotoelektrischen Effekt.\\
Mit dieser Schaltung kann man also sicherstellen, dass der Fotowiderstand immer dann einen bestimmten Lichteinfall misst, wenn kein Strom gemessen wird.

\begin{figure}[!htb]
	\centering	
	\includegraphics[scale=0.7]{Brueckenschaltung.png}
	\caption{Wheatstone'sche Brückenschaltung \cite{lp}}
	\label{fig:bruecke}
\end{figure}


\section{Durchführung}
\label{sec:durchfuehrung}
\subsection{Versuchsaufbau}
\begin{figure}[!htb]
	\centering	
	\includegraphics[scale=0.7]{Aufbau_schematisch.png}
	\caption{schematischer Versuchsaufbau \cite{lp}}
\end{figure}
Zuerst wird der Spalt so justiert, dass auf den Photowiderstand die maximale Intensität trifft.
Außerdem benötigt man noch zwei Stoppuhren für die späteren Messungen und die Wheatstone'sche Brücke muss verkabelt werden.

\subsection{Konzentrationsverlauf in Abhängigkeit der Zeit}
Für den Graufilter $c_0/16$ regelt man das Potentiometer so, dass das Amperemeter keinen Strom anzeigt.
Dann wird die Küvette zu $3/4$ mit Wasser gefüllt, darüber Methylenblau.
Man startet die Stoppuhr, nachdem man die Küvette in den Strahlengang gestellt hat.
Für eine halbe Stunde notiert man alle 30 Sekunden den Ort der Konzentration $c_0/16$.
Dabei wird jedoch die Messbrücke nicht verändert, sondern die Küvette mittels Micrometerschraube nach oben bewegt, bis das Amperemeter wieder keinen Strom zeigt.\\
Nach der Messung wird diese Küvette vorsichtig zur Seite gestellt, damit sich die Flüssigkeiten nicht zusätzlich vermischen.
Die benutzte Stoppuhr lässt man für eine spätere Messung weiterlaufen.
Dann füllt man eine zweite Küvette wie die erste, gleicht aber die Messbrücke mit dem Graufilter $c_0/32$ ab.
Die vorige Messung wird mit der zweiten Küvette, der zweiten Stoppuhr und dem anderen Filter wiederholt.
\subsection{Konzentrationsprofil}
Etwa 40 Minuten nach Beginn der letzten Messung wird die Konzentrationsverteilug der zweiten Küvette in Abhängigkeit des Ortes gemessen.
Dies sollte schnell geschehen, damit die Zeit als konstant angenommen werden kann.
Trotzdem notiert man Beginn und Ende dieser Messung, die Stoppuhr muss also weiter laufen.
Der Messvorgang sieht folgendermaßen aus:
Nacheinander wird die Messbrücke auf die verschiedenen Graufilter $c_0/2, c_0/4, c_0/8, c_0/16, c_0/32$ abgeglichen, bevor man dazu die Stelle sucht, an der genau diese Konzentration herrscht.
Diesen Vorgang wiederholt man nochmal in umgekehrter Reihenfolge der Graufilter.\\
Diesen ganzen Messvorgang wird für die erste Küvette nach ca. 100 Minuten  seit Beginn der zugehörigen ersten Messung wiederholt. 

\section{Auswertung}
\label{sec:auswertung}

\subsection{erwartete Diffusionskurven}
\begin{figure}
	\input{errorfkt.tex}
	\caption{Profil für den Diffusionskoeffizienten $D=4\cdot 10^{-10} ~ \si{\meter ^2 / \second}$ zu verschiedenen Zeiten}
	\label{fig:erwDiffKurve}
\end{figure}

Aus der Lösung der Diffusionsgleichung \eqref{eq:DiffLsg} ergibt sich Abbildung \ref{fig:erwDiffKurve}.
Man erkennt,dass es zu $t=0$ eine klare Grenze zwischen den beiden Flüssigkeiten gibt.
Im Verlauf der Zeit (hier nach 30 und 90 Minuten bzw. 6 Stunden und 2 Tagen) vermischen sie sich und die Konzentration nähert sich überall $c_0/2$ an.
Bei $x=0$ befindet sich jedoch immer die Konzentration $c_0/2$.\\
Bei den ersten beiden Messungen wird der $c/c_0$ Wert konstant gehalten und der Ort dieser Konzentration im Verlaufe der Zeit verfolgt.
Also in Abb. \ref{fig:erwDiffKurve} die Schnittpunkte einer parallelen Gerade zur x-Achse mit den Kurven zu unterschiedlichen Zeitpunkten.\\
In den letzten beiden Messungen verfolgt man Werte auf einer Kurve mit konstanter Zeit.

\subsection{Berechnung des Diffusionskoeefizienten}
\begin{align}
	\erf(y)=\erf\left(\frac{x}{\sqrt{4Dt}}\right)
\end{align}

\begin{align}
	D&=\frac{m}{4~C^2}\\
	\sigma_D&=\frac{\sigma_m}{4~C^2}
\end{align}

\begin{figure}
	\input{DiffStrecke.tex}
	\caption{???????????????????}
	\label{fig:mess12}
\end{figure}

Für Messung 1 und 2 ergeben sich die Werte aus Tabelle \ref{tab:ausw12}.
\begin{table}[!htb]
\centering
\begin{tabular}{|c|c|c|}
	\hline		
	& Messung 1 & Messung 2 \\
	& $c_0/16$ & $c_0/32$ \\
	\hline
	\hline
	C genähert \footnotemark & $1.085$ & $1.317$ \\	
	
	Steigung m & 
	$(18.8 \pm 0.3) \cdot 10^{-10}\si{\meter ^2 / \second}$ & 
	$(19.7 \pm 0.6) \cdot 10^{-10} \si{ \meter ^2 / \second}$ \\	
	
	Diffusionskoeffizient D &
	$(4.00 \pm 0.06) \cdot 10^{-10}\si{\meter ^2 / \second}$ & 
	$(2.83 \pm 0.09) \cdot 10^{-10} \si{ \meter ^2 / \second}$ \\	
	\hline		
\end{tabular}
\caption{Auswertung Messung 1 und 2}
\label{tab:ausw12}
\end{table}
\footnotetext{\textit{WolframAlpha}, www.wolframalpha.com}

Daraus kann der gewichtete Mittelwert des Diffusionskoeffizienten berechnet werden
\begin{empheq}[box=\shadowbox*]{align}
\overline{D}&=(3.63\pm 0.04) \cdot 10^{-10}\si{ \meter ^2 / \second}
\end{empheq}

\subsection{Konzentrationsprofil}
\begin{figure}[!htb]
	\input{Diffprofil40.tex}
	\caption{Messung 3}
\end{figure}

\begin{figure}[!htb]
	\input{Diffprofil100.tex}
	\caption{Messung 4}
\end{figure}

\section{Diskussion}
\label{sec:diskussion}
\subsection{Berechnung des Diffusionskoeefizienten}
Betrachtet man Abb. \ref{fig:mess12}, fällt auf, dass sich bei der ersten Messung $c_0/16$ wie erwartet eine Gerade ergibt. Bei der zweiten Messung ergibt sich jedoch eher etwas quadratisches.
Deshalb sind beide Ursprungsgeraden ungefähr gleich und erklärt den großen Unterschied zwischen den beiden Werten für $D$.
Dies ergibt physikalisch wenig Sinn.
Daher muss nach einem Fehler in der Messung gesucht werden.
Zum einen kann es daran liegen, dass wir bei der zweiten Messung zu lange zwischen Befüllen und Messen gewartet haben und sich die Flüssigkeiten schon vor der Messung vermischt haben.
Dies geschieht auch, wenn die Küvette geschüttelt wurde.
Berücksichtigt man dies,...............

\subsubsection{Filter}
Außerdem haben wir vergessen, die unterschiedlichen Filter zu reinigen.
So hat sich der Wert geändert, wenn man die Höhe verändert hat, in der sich der Filter befand.
Wir können also nicht garantieren, dass für die Filter die richtigen Werte gemessen wurden.
Da uns dies aber zur Messzeit auffiel, haben wir zusätzlich noch die Potentiometer-Einstellung sowie den Wert des dritten Widerstandes notiert.
Aus den Daten kann man ablesen,............

\section{Anhang}

\begin{thebibliography}{100}

\bibitem{lp} 
	\emph{Lehrportal der Universität Göttingen, Diffusion},
  http://lp.uni-goettingen.de/get/text/3665, abgerufen 09.07.14 11:21 Uhr

\bibitem{gerthsen}
	\textsc{Dieter Meschede} (2010): \emph{Gerthsen Physik}, 24. Auflage, Springer Heidelberg
Dordrecht London New York

\end{thebibliography}

\end{document}
